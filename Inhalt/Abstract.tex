\chapter*{Abstract}
\thispagestyle{empty}
An abstract comprises a one-paragraph summary of the whole paper.
Abstracts have become increasingly important, as electronic publication databases
are the primary means of finding research reports in a certain subject area
today \cite{Koopman1997}.

According to \cite{Day2012}, there are two basic types of abstract:
\begin{itemize}
  \item An \textit{informative} abstract extracts everything relevant from the
  paper, such as research objectives addressed, methods employed in solving the problems,
results obtained, and conclusions drawn. Such abstracts may serve as a highly
aggregated substitute for the full paper.
\item An \textit{indicative or descriptive} abstract rather describes the
content of the paper and may thus serve as an outline of what is presented in
the paper. This kind of abstract cannot serve as a substitute for the full text.
\end{itemize}
A checklist defining relevant parts of an abstract is proposed in
\cite{Koopman1997}:
\begin{itemize}
\item \uline{Motivation:} Why do we care about the problem and the results?
\item \uline{Problem:} What problem is the paper trying to solve and what is
the scope of the work?
\item \uline{Solution:} What was done to solve the problem?
\item \uline{Results:} What is the answer to the problem?
\item \uline{Conclusions:} What conclusions does the answer imply?
\end{itemize}
Also as suggested in \cite{Day2012} there are few things that should not be
included in an abstract, i.e. information and conclusions not stated in the
paper, the exact title phrase, and illustrative elements such as tables and figures.  It is also
not beneficial to use the exact phrases that appear later in the introduction.