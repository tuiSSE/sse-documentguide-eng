%%%%%%%%%%%%%%%%%%%%%%%%%%%%%%%%%%%%%%%%%%%%%%%%%%%%%%%%%%%%%%%%%%%%%%%
%% 
%%%%%%%%%%%%%%%%%%%%%%%%%%%%%%%%%%%%%%%%%%%%%%%%%%%%%%%%%%%%%%%%%%%%%%%
\chapter{Introduction}

In the Introduction, you are attempting to inform the reader about the rationale
behind the work, justifying why your work is an essential component of research
in the field. The introduction gives an overall review of the paper. Much  of 
the Introduction  should  be  written  in  the  present  tense, because  the  problem  and  the established  knowledge relating  to  it  at  the  start
of  the  work will  be primarily referred.

As recommended in \cite{Majid1993}, the Introduction should be structured as
follows:

 \uline {Motivation and Background Information.}
 This should recall to the reader why the kind of result mentioned already in
 the abstract would be interesting and important. 
 It states the motivation, so if the reader agrees with the way you are
 looking at the field, there's some probability that the paper will be useful
 for them. Also purpose in writing the paper should be clearly stated.
 Sufficient  background  information should be supplied to  allow the  reader  
to understand  and  evaluate the  results  of  the  present study.
\par  \uline{The results and strategy.}
 It should present first, the nature and scope of the  problem  investigated.
 It should state the method and principal results of the investigation
 \cite{Day2012}. Reference the previous work which inspired and led up to your
 result. A good way is to tell a story, an interesting one that puts everything
 into perspective are the existing literature and conveys how it is you succeeded where others
failed. What was the key idea which nobody else spotted? The reader should not
be kept in suspense; instead allowed to follow the development of the evidence.
 
\par \uline{Outline the organisation.}
This should be brief but not simply a list. State the goal and main achievement
of each section. Make it into a story whereby each section is logically a
precursor to the next section. 

The aim of this report is to guide in creating a plan that motivates original thinking and
 to only do your own best. It also provides information on how to work on
 a research document. It covers the entire process which includes
 gathering requirements, searching for information about the topic,
 proposing working answers and supporting them and writing the research
 document.

