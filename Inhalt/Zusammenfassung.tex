\chapter{Summary and Outlook}

As stated in \cite{Booth2003}, same elements as in Introduction can be used for
conclusion but in reverse order.
\begin{itemize}
\item Start with Your Main Point
State the main point again at the beginning of your conclusion, but state it
more fully. It should not simply repeat the introduction.
\item Add a New Significance or Application
Point to a new significance of the claim which is not mentioned in the
introduction. But care should be taken not to broaden a possible significance 
so much that it seems as the main point. 
\item Add a Call for More Research
More research still to do can be called at the end of the conclusion. That keeps
the conversation alive. Those who pursue your suggestion will review your work, respond to it, and move beyond it.
\end{itemize} 
